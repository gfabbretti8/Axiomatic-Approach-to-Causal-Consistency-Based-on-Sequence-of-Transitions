

\begin{definition}[Parabolic Lemma]
  For any path $r$ there exist paths $r',r''$ and forward only paths $s,s'$ such that $r'rr'' \approx \rev{s}s'$.
\end{definition}

\begin{proposition}
  Suppose an $LTSI$ satisfies both $SP$ and $BTI$. Then $PL$ holds.
\end{proposition}
\begin{proof}
  Suppose $BTI$ and $SP$ hold. Define a function on path as follows: d(s) is the
  number of pairs $(\sigma_1,\sigma_2)$ such that $\sigma_1$ occurs to the left of $\rev{\sigma_2}$ in $s$.
  Clearly $s$ is parabolic iff $d(s)=0$. \\

  Suppose $d(s)>0$. We show that there is $s' \approx s$ and $d(s') < d(s)$.
  \begin{description}
  \item[Case $d(s)>0$] Now there are two cases to consider.
    \begin{description}
    \item[Case $t = u$] Then $\delta_1 t\rev{u}\delta_2 \approx
      \delta_1\delta_2$.
    \item[Case $t \neq u$] In this case thanks to $BTI$ we know that there are
      two cases to consider.
      \begin{description}
      \item[Case $\rev{t} \ind \rev{u}$] Then by $SP$ we have $\rev{t}\rev{u'}
        \approx \rev{u}\rev{t'}$. Hence $s= \sigma_1t\rev{u}\sigma_2 \approx
        \sigma_1 t\rev{u}\rev{t'}t'\sigma_2 \approx \sigma_1
        t\rev{t}\rev{u'}t'\sigma_2 \approx \sigma_1 \sigma_2=s'$.
        
      \item[Case $t \not\ind \rev{u}$] In this case thanks to $BTI$ we know that
        there exist $t_c$ and $\rev{u_c}$ s.t. $t_ct \ind \rev{uu_c}$. Then by
        $SP$ we can apply the same reasoning as the previous case.
        
      \end{description}
    \end{description}
  \end{description}
  
\end{proof}

\begin{definition}\label{def:cc}
  {\bf Causal Consistency (CC)}: if $r$ and $s$ are coinitial and cofinal paths then $r \ceqt s$.
\end{definition}

\begin{proposition}\label{prop:PL WF CC}
  Suppose an LTSI satisfies WF and PL. 
  Then CC holds.
\end{proposition}
\begin{proof}
  Let $r:P \ptran \rho Q$ and $r':P \ptran {\rho'} Q$.
  Using WF, let $I,s$ be such that $s:I \ptran \sigma P$, $I \in \Irr$.
  Now $sr\rev{sr'}$ is a path from $I$ to $I$,
  and so by PL there are $r_1,r_2$ forward-only such that $\rev{r_1}r_2 \ceqt sr\rev{sr'}$.
  But $I \in \Irr$ and so $r_1 = \es$ and $r_2 = \es$.
  Thus $\es \ceqt sr\rev{sr'}$, so that $sr \ceqt sr'$ and $r \ceqt r'$
  as required.
\end{proof}